\documentclass{article}

\usepackage[margin=1in]{geometry}
\usepackage{tikz}
    \tikzstyle{hollow node} = [circle,draw,inner sep=1.5]
    \tikzstyle{solid node}  = [circle,draw,inner sep=1.5,fill=black]
    \usetikzlibrary{calc}
\usepackage{sgame}
    \gamemathtrue
\usepackage[shortlabels]{enumitem}
\usepackage{amsmath}

\author{Damien Prieur}
\title{Problem Set 6 \\ ECON 250}
\date{}

\begin{document}

\maketitle

\section{\emph{Ultra} and \emph{Soar}}
Two startups, \emph{Ultra} and \emph{Soar}, have developed fairly similar ridesharing apps and are competing for customers by deciding how much to advertise their app.
Let $A_{i} \ge 0$ be the advertising level of firm $i$.
The profits of each firm depend on how much it advertises its app but also on the advertising level of its opponent $A_{-i}$.
The higher the advertising levels the more visibility ridesharing apps and demand increases.
In particular, firm $i$'s profits are $u_{i}(A_{i},A_{-i}) = (90 + A_{i} + A_{-i})A_{i}-2A_{i}^{2}$.
\emph{Ultra} chooses $A_{1}$ first and, after observing it, \emph{Soar} chooses $A_{2}$.
\begin{enumerate}[(a)]
\item Draw the extensive form. Which are the proper subgames?

\item Find the SPNE. How does it compare to the NE of the simultaneous move game?

\item Is there a first-mover advantage in the market? Justify your answer.

\end{enumerate}

\section{\emph{Do it now or do it later}}
\emph{Do it now or do it later}. My friend Paco Vivalavida must do an activity \emph{only once} between now ($t = 1$) and $T = 4$.
That is, he has to choose a vector of decisions $(x_1,x_2,x_3,x_4), x_t \in {Y,N}$, where $x_t = Y$ means do the activity in period $t$ if he has not already don it at $t'<t$, and $x_t = N$ means not do the activity at $t$.
For instance, $(N,Y,N,Y)$ means Paco does not do the activity at $t = 1$, does the activity at $t = 2$ if it was not done before, does no do it at $t = 3$ if it has not been done before and does it at $t=4$ if it has not been done before.
Such a strategy leads to him actually doing the activity at $t = 2$.

\begin{enumerate}
\item \emph{Rewarding Activity}. Paco loves movies and he just got a free movie ticket that he can use at the local movie theater in one of the next 4 Saturdays.
Each Saturday a different movie is shown, with better movies scheduled at later dates. In particular, Paco's (undiscounted) instantaneous utilities associated to each of the moves are $(u_1, u_2, u_2, u_4) = (3, 5, 8, 13)$, while he gets 0 if he does not watch a movie at $t$.
He has $\beta - \delta$ time preferences with $\delta=1$.
For instance, his utility at $t = 1$ from $(N,Y,N,Y)$ is $\beta u_2$ because he watches the 2nd-week movie, while her period-2 self derives utility $u_2$.
Similarly, his period-1 self gets $\beta u_4$ when he chooses $(N,N,N,Y)$ while her period-3 self gets $\beta u_4$.
[Hint: A useful way to visualize the decision problem is to draw a game tree in which an action $Y$ at period $t$ leads to a terminal node with payoff $u_t$, while an action $N$ leads to the next period decision node, except in the last period in which only action $y$ is available since he has to do the activity if he has not done so before.]
\begin{enumerate}[(a)]
\item If Paco's time preferences are given by $\beta = 1$, what is his optimal choice of $(x_1,x_2,x_3,x_4)$?
\newline

\item For the remainder of the problem assume Paco is present-biased with $\beta = 0.5$.
If he could commit to $(x_1, x_2, x_3, x_4)$ (e.g. by giving the coupon to a friend and instructing him to deliver it at a specific $t$), what would his choice be?
\newline

\item If Paco is sophisticated, what would he choose?
Does he 'preproperate' (do an activity too soon)?
[Hint: draw the game tree and use backward induction, applying the discounting as we did in the alternating offers bargaining game.]
\newline
\end{enumerate}
\end{enumerate}

\end{document}
