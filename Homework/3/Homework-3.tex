\documentclass{article}

\usepackage[margin=1in]{geometry}
\usepackage{tikz}
    \tikzstyle{hollow node} = [circle,draw,inner sep=1.5]
    \tikzstyle{solid node}  = [circle,draw,inner sep=1.5,fill=black]
    \usetikzlibrary{calc}
\usepackage{sgame}
    \gamemathtrue
\usepackage[shortlabels]{enumitem}
\usepackage{amsmath}

\author{Damien Prieur}
\title{Problem Set 2 \\ ECON 250}
\date{}

\begin{document}

\maketitle

\section{Differentiated Bertrand Competition}
\emph{Differentiated Bertrand Competition}.
There are two firms $i = 1, 2$ simultaneously choosing prices $p_{i} \in [0,1]$.
The demand of firm $i$ is $(D_{i}(p_{i},p_{-i})= 1-3p_{i} + p_{-i}$ and it has zero production costs.
That is, firm $i$'s payoff is $p_{i}D_{i}(p_{i},p_{-i})$.

\begin{enumerate}[(a)]

\item Find the best response function of firm $i$.
\newline
$$u_{i}(p_{i},p_{-i}) = p_{i} ( 1 - 3p_{i} + p_{-i})$$
$$\frac{\partial u_{i}}{\partial p_{i}} = p_{i}(-3) + (1-3p_{i}+p_{-i}) = -6p_{i} + p_{-i} + 1$$
Setting equal to zero to find the maximum we get
$$ \frac{\partial u_{i}}{\partial p_{i}} = 0 \Rightarrow -6p_{i} + p_{-i} + 1 = 0 \Rightarrow p_{i} = \frac{p_{-i} + 1}{6}$$
So for each player we have
$$ BR_{1}(p_2) = \frac{p_2+1}{6} \qquad BR_{2}(p_1) = \frac{p_1+1}{6}$$

\item Find the set of 1-rationalizable, 2-rationalizable, and 3-rationalizable strategies.
\newline
1-rationalizable - best response to some $\sigma_{-i}$.
Best response range for all inputs = $BR_{i}(0) = \frac{1}{6}$ and $BR_{i}(1) = \frac{1}{3}$.
Since the $BR$ function is linear then we know that the range of 1-rationalizable strategies is $p_{i}\in[\frac{1}{6}, \frac{1}{3}]$
\newline
Repeating this same logic but with the 1-rationalizable strategies as the domain we get $BR_{i}(\frac{1}{6}) = \frac{7}{36}$ and $BR_{i}(\frac{1}{3}) = \frac{2}{9}$.
So our range of 2-rationalizable strategies is $p_{i} \in [\frac{7}{36},\frac{2}{9}]$.
\newline
Doing one more substitution we get $BR_{i}(\frac{7}{36}) = \frac{43}{216}$ and $BR_{i}(\frac{2}{9}) = \frac{11}{54}$.
So our range of 3-rationalizable strategies is $p_{i} \in [\frac{43}{216},\frac{11}{54}]$.

\item Find the unique PSNE.
\newline
When are both players best responding to each other? \\
$$ BR_{i}(BR_{-i}(p_{i})) = p_{i} \Rightarrow \frac{\frac{p_{i} + 1}{6} + 1}{6} = p_{i} $$
$$\frac{p_{i} + 7}{6} = 6p_{i} \Rightarrow p_{i} = \frac{7}{35} = \frac{1}{5} = 0.2$$
The unique PSNE for this is for both players to play $0.2$ giving a strategy profile $(.2,.2)$.
\end{enumerate}

\section{Guessing Game}
Consider the following Guessing Game.
There are $n = 10$ players simultaneously choosing a number in $\{1,2,3\}$.
The winners are the closest to $\frac{1}{2}$ the average guess (they even split the prize between the winners if there is more than one).
Find the set of rationalizable strategy profiles. Justify your answer.
\newline
\newline
It is never rational to pick 3 as your number. If everyone picks 3 then the average will be 3 as well, and the winner will be the player closest to 1.5.
Therefore you could get the whole prize if you guessed two if everyone guessed 3 and should never pick 3 as your guess.
By the same logic if everyone is aware that 3 is strategy that never wins  everyone will choose between $\{1,2\}$.
Applying the same logic as before 2 will never be picked as picking 1 strictly dominates 2.
The set of rationalizable strategy profiles is everyone chooses 1 which causes the payout to be split evenly between everyone.

\end{document}
