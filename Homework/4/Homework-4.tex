\documentclass{article}

\usepackage[margin=1in]{geometry}
\usepackage{tikz}
    \tikzstyle{hollow node} = [circle,draw,inner sep=1.5]
    \tikzstyle{solid node}  = [circle,draw,inner sep=1.5,fill=black]
    \usetikzlibrary{calc}
\usepackage{sgame}
    \gamemathtrue
\usepackage[shortlabels]{enumitem}
\usepackage{amsmath}

\author{Damien Prieur}
\title{Problem Set 4 \\ ECON 250}
\date{}

\begin{document}

\maketitle

\section{WATSON Chapter 11, Exercise 6}
Determine all of the Nash Equilibiria (pure-strategy and mixed-strategy equilibira) of the following games.

\begin{enumerate}[(a)]
\item
\begin{game}{2}{2}[Player 1][Player 2]
    &    H    &    T    \\
H   &   1,-1  &   -1,1  \\
T   &   -1,1  &   1,-1  \\
\end{game}
\newline
PSNE: None, game of pure conflict \\
MSNE: $u_{1}((p,1-p), H) = p - (1-p)$ and $u_{1}((p, 1-p), T) = -p + (1 - p) \implies 2p-1 = -2p + 1 \implies p = \frac{1}{2}$
Due to the symmetry of the game player 2 will have the same result which gives us a mixed strategy nash equilibrium with $\sigma_{1}=(.5,.5)$ and $\sigma_{2}=(.5,.5)$.


\item
\begin{game}{2}{2}[Player 1][Player 2]
    &    C    &    D    \\
C   &   2,2   &   0,3   \\
D   &   3,0   &   1,1   \\
\end{game}
\newline
PSNE: (D, D) \\
MSNE:
$$ u_{1}((p,1-p), C) = u_{1}((p,1-p), D) \implies 2p + 3(1-p) = 0p + 1(1-p) \implies -p + 3 = 1 - p \implies 3=1$$
There is no $p$ that makes this true since both player will always pick D and never pick C.
So there is no MSNE.

\item
\begin{game}{2}{2}[Player 1][Player 2]
    &    H    &    D    \\
H   &   2,2   &   3,1   \\
D   &   3,1   &   2,2   \\
\end{game}
\newline
PSNE: None\\
MSNE: \\
$$ u_{1}((p,1-p), H) = u_{1}((p,1-p), D) \implies 2p + 3(1-p) = 3p + 2(1-p) \implies -p + 3 = p + 2 \implies p = \frac{1}{2}$$
$$ u_{2}(H, (p,1-p)) = u_{2}(D, (p,1-p)) \implies 2p + (1-p) = p + 2(1-p) \implies p + 1 = 2 -p \implies p = \frac{1}{2}$$
Which gives us a MSNE of $\sigma_{1} = (.5,.5)$ and $\sigma_{2} = (.5, .5)$

\item
\begin{game}{2}{2}[Player 1][Player 2]
    &    A    &    B    \\
A   &   1,4   &   2,0   \\
B   &   0,8   &   3,9   \\
\end{game}
\newline
PSNE: (A, A), (B,B) \\
MSNE: \\
$$ u_{1}((p,1-p), A) = u_{1}((p,1-p), B) \implies p + 0(1-p) = 2p + 3(1-p) \implies p = -p + 3 \implies p = \frac{3}{2}$$
$$ u_{2}(A, (p,1-p)) = u_{2}(B, (p,1-p)) \implies 4p + 0(1-p) = 8p + 9(1-p) \implies 4p = 9 - p \implies p = \frac{9}{5}$$
Which gives us a MSNE of $\sigma_{1} = (1.5,-.5)$ and $\sigma_{2} = (1.8, -.8)$ \\
Solution doesn't make sense since both equations don't have a solution where $ p \in [0,1]$

\item
\begin{game}{2}{2}[Player 1][Player 2]
    &    A    &    B    \\
A   &   2,2   &   0,0   \\
B   &   0,0   &   3,4   \\
\end{game}
PSNE: (A, A), (B,B) \\
MSNE: \\
$$ u_{1}((p,1-p), A) = u_{1}((p,1-p), B) \implies 2p + 0(1-p) = 0p + 3(1-p) \implies 2p = 3 - 3p \implies p = \frac{3}{5}$$
$$ u_{2}(A, (p,1-p)) = u_{2}(B, (p,1-p)) \implies 2p + 0(1-p) = 0p + 4(1-p) \implies 2p = 4 - 4p \implies p = \frac{2}{3}$$
Which gives us a MSNE of $\sigma_{1} = (\frac{3}{5},\frac{2}{5})$ and $\sigma_{2} = (\frac{2}{3}, \frac{1}{3})$

\item
\begin{game}{3}{3}[Player 1][Player 2]
    &    L    &    M    &    R    \\
U   &   8,1   &   0,2   &   4,3   \\
C   &   3,1   &   4,4   &   0,0   \\
D   &   5,0   &   3,3   &   1,4   \\
\end{game}
\newline
L is strictly dominated by M and can be removed.
\newline
\begin{game}{3}{2}[Player 1][Player 2]
    &    M    &    R    \\
U   &   0,2   &   4,3   \\
C   &   4,4   &   0,0   \\
D   &   3,3   &   1,4   \\
\end{game}
\newline
C is dominated by $\sigma_{1} = (.75, 0, .25) $ and can be removed.
PSNE: (C, M), (U,R) \\
MSNE: \\
$$ u_{1}((p, q, 1-p-q), M)  = u_{1}((p, q, 1-p-q), R) \implies 4q + 3(1-p-q) = 4p + (1 - p - q) \implies 3 + q -3p = 1 + 3p -q$$
$$ p = \frac{1+q}{3} $$
$$ u_{2}(U, (p,1-p)) = u_{2}(C, (p,1-p)) = u_{2}(D, (p,1-p)) \implies 2p + 3(1-p) = 4p + 0(1-p) = 3p + 4(1-p) $$
$$\implies 3-p = 4p = 4-p \implies p = \frac{4}{5} \quad \text{or} \quad p = \frac{3}{5}$$
Which gives us MSNE of $\sigma_{1} = (\frac{1+q}{3}, q, 1 - \frac{1+q}{3} - q) \quad q \in [0,1]$ and $\sigma_{2} = (0, \frac{4}{5}, \frac{1}{5}) \text{or} (0, \frac{3}{5}, \frac{2}{5})$

\end{enumerate}

\end{document}
