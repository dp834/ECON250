\documentclass{article}

\usepackage[margin=1in]{geometry}
\usepackage{sgame}
    \gamemathtrue
\usepackage[shortlabels]{enumitem}
\usepackage{amsmath}

\author{Damien Prieur}
\title{Problem Set 7 \\ ECON 250}
\date{}

\begin{document}

\maketitle

\section*{Question 1}
Two players play a repeated game in which they play the following stage game twice, and discount payoffs by $0 < \delta < 1$.
\begin{center}
\begin{game}{2}{2}
    &  $h$  &  $d$  \\
$H$ &  0,0  &  3,1  \\
$D$ &  1,2  &  2,2  \\
\end{game}
\end{center}

\begin{enumerate}[(a)]
\item Find all the SPNE in non-contingent strategies. Focus only on pure strategies.
\newline
\newline
First we must find the PSNE at each stage of the game.
In our case we have $\{(D,h), (H,d)\}$.
Each of the SPNE is a combination of playing these PSNE regardless of history.

\begin{enumerate}[1.]
\item $s_1(h^t) = D \text{ and } s_2(h^t) = h $
\item $s_1(h^t) = H \text{ and } s_2(h^t) = d $
\end{enumerate}

\item Find the conditions under which playing $(D,d)$ can be part of a SPNE.
\newline
\newline
Attempt a grim trigger strategy $s_2(h^1) = d$ and $s_2(h^2) = h$ if $h^2 = Hd$  and $s_2(h^2) = d$ if $h^2 \neq Hd$.
\newline
Payoff from player 1 playing $H$ at $t=1:$ at most $3 + \delta(1)$
\newline
Payoff from player 1 playing $D$ at $t=1:$ at most $2 + \delta(3)$
\newline
If $\delta \geq 0.5$ then $(D,d)$ is part of a SPNE if
\newline
$s_1(h^1) = D \text{ and } s_1(h^2) = H$ and $s_2(h^1) = d$ and $s_2(h^2) = h$ if $h^2 = Hd$  and $s_2(h^2) = d$ if $h^2 \neq Hd$.



\end{enumerate}
\end{document}
